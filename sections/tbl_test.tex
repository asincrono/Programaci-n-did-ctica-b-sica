%\documentclass[a4paper,twoside,titlepage]{scrartcl}
\documentclass[a4paper,twoside,titlepage,12pt]{article}
% NOTA: No olvidar incluir "/" al final de la ruta.
\usepackage{import}

\usepackage[a4paper, driver=xetex, left=2cm, right=1.5cm, top=2cm, bottom=2cm]{geometry}
\usepackage{polyglossia}
\setmainlanguage{spanish}

\usepackage{fontspec}
\setmainfont{Arial}

% \usepackage{xcolor}
% \colorlet{borrador}{red}

\usepackage{float}

\usepackage{rotating}

\usepackage{longtable}
%\usepackage{tabularx}
% ltablex combina las características de tabularx y longtable.
\usepackage{ltablex}
\keepXColumns % Evitar que el ancho indicado en tabularx se tome como máximo.
% \usepackage{tabulary}
% \usepackage{tabu}
\usepackage{booktabs}
\usepackage{multirow}
\usepackage{makecell}

\newcolumntype{M}[1]{>{\centering\arraybackslash}m{#1}}
\renewcommand{\tabularxcolumn}[1]{>{\small}m{#1}}

\begin{document}
    \begin{tabularx}{0.5\linewidth}{p{2cm} X X p{2cm}}
    \toprule
    Number & Multiline par. & Multiline par. & Symbol \\ \midrule
    1 & La razón de la sinrazón que a mi razón se hace & De tal manera mi razón enflaquece & X\\
    2 & La razón de la sinrazón que a mi razón se hace & De tal manera mi razón enflaquece & X\\ 
    \bottomrule
    \end{tabularx}


% \begin{longtabu} to 0.75\textwidth {X[c,m] X X X[3,c,m]}
%     \toprule
%     Number & Paragraph & Another paragraph & Check \\ \midrule
%     \endfirsthead
%     \endhead
%     1 & La razón de la sinrazón que a mi razón se hace & De tal manera mi razón enflaquece & X \\
%     2 & La razón de la sinrazón que a mi razón se hace & De tal manera mi razón enflaquece &  \\
%     3 & La razón de la sinrazón que a mi razón se hace & De tal manera mi razón enflaquece & X \\
%     \bottomrule
% \end{longtabu}



% \begin{longtabu} to 0.5\textwidth {X X X[2,c,m] X[2,c,m]}
%     \toprule
%     \multicolumn{2}{c}{Group of columns} & \\ \cmidrule{1-2}
%     Col. a & Col. b & \multirow{-2}{*}{Thick col. c} & \multirow{-2}{*}{Thick col. d} \\ \midrule
%     \endfirsthead
%     \endhead
%     Aaaa & Bbbb & Cccc & Dddd \\ \bottomrule 
% \end{longtabu}


% \begin{longtabu} to 0.5\textwidth {X X X[2,c,m] X[2,c,m]}
%     \toprule
%     \multicolumn{2}{c}{Group of columns} & \\ \cmidrule{1-2}
%     Col. a & Col. b & \multirow{-2}{*}{Thick col. c} & \multirow{-2}{*}{Thick col. d} \\ \midrule
%     \endfirsthead
%     \endhead
%     Vivir así es morir de amor & Por amor tengo el alma heriiiiiida & Melancolía & Melancolía \\ \bottomrule 
% \end{longtabu}

\newpage
% \renewcommand{\tabularxcolumn}[1]{>{\small}m{#1}}
\begin{tabularx}{0.5\linewidth}{M{2cm} X X M{2cm}}
    \toprule
    Number & Multiline par. & Multiline par. & Symbol \\ \midrule
    1 & La razón de la sinraón que a mi razón se hace & De tal manera mi razón enflaquece & X\\
    2 & La razón de la sinraón que a mi razón se hace & De tal manera mi razón enflaquece & X\\ 
    \bottomrule
\end{tabularx}

% %\begin{sidewaystable}[!htbp]

%     \begin{tabularx}{\textwidth}{| P{0.9cm} | P{0.9cm} | P{0.9cm} | P{0.9cm} | P{0.9cm} | P{0.9cm} | P{1cm} | X | P{0.5cm} | P{1cm} |}
%         \caption{Unidades didácticas y su duración en relación a los R.A.  incluidos, indicando también su duración (D.) y peso (P.).}
%         \label{tab:ud_ra}\\\hline
        
%         \multicolumn{6}{| c |}{Resultados de aprendizaje} &  & \multirow{2}{=}{Título y descripción} & \multirow{2}{=}{D.} & \multirow{2}{=}{P.} \\\cline{1-6}
%         RA1 & RA2 & RA3 & RA4 & RA5 & RA6 & \multirow{-2}{*}{U.D.} &  &  &  \\\hline
%         \endfirsthead
%         \hline
%         RA1 & RA2 & RA3 & RA4 & RA5 & RA6 & U.D. & Título y descripción & D. & P. \\\hline
%         \endhead
        
%         \multicolumn{6}{| c |}{} &  & Total &  & 100\% \\\hline
%         \endlastfoot
        
%         X & & & & & & 1 & \makecell[l{X}]{\textbf{Fundamentos:}\\ Elementos fundamentales de un sistema operativo} & ? & 20\% \\\hline
        
%         & X & & & & & 3 & \makecell[l{X}]{\textbf{Instalación:}\\ Proceso de instalación de varios sistemas operativos monousuario.} & 5 & 15\% \\\hline
        
%         & & X & & & & 4 & \makecell[l{X}]{\textbf{Configuración:}\\ Configuración del sistema operativo adecuándolo a las necesidades del usuario en integrándolo en el entorno de trabajo.} & 5 & 20\% \\\hline
        
%         & & & X & & & 5 & \makecell[l{X}]{\textbf{Administración y optimización:}\\ Donde trataremos de las tareas de administración y métodos de análisis de carga para la optimización del sistema.} & 5 & 10\% \\\hline
        
%         & & & & X & & 6 & \makecell[l{X}]{\textbf{Herramientas de linea de comandos:}\\ Donde veremos el funcionamiento de la linea de comandos la sus herramientas administrativas principales.} & 5& 15\% \\\hline
        
%         & & & & & X & 2 & \makecell[l{X}]{\textbf{Virtualización:}\\ Describimos las características y utilidad de las máquinas virtuales. Crearemos varias máquinas virtuales en las que instalaremos distintos sistemas operativos.} & 5& 20\% \\\hline
%     \end{tabularx}
%     %\end{sidewaystable}

%     \newpage

\begin{tabularx}{\textwidth}{c X c}
    \thead{Uno} & \thead{Dos} & \thead{Tres}\\ 
    1 & 2 & 3 \\
\end{tabularx}

\end{document}