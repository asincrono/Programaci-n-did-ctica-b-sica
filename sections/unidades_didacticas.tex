\section{Unidades didácticas}

%\begin{sidewaystable}[!htbp]

    \begin{tabularx}{\textwidth}{| P{0.9cm} | P{0.9cm} | P{0.9cm} | P{0.9cm} | P{0.9cm} | P{0.9cm} | P{1cm} | X | P{0.5cm} | P{1cm} |}
    \caption{Unidades didácticas y su duración en relación a los R.A.  incluidos, indicando también su duración (D.) y peso (P.).}
    \label{tab:ud_ra}\\\hline
    
    \multicolumn{6}{| c |}{Resultados de aprendizaje} &  & \multirow{2}{=}{Título y descripción} & \multirow{2}{=}{D.} & \multirow{2}{=}{P.} \\\cline{1-6}
    RA1 & RA2 & RA3 & RA4 & RA5 & RA6 & \multirow{-2}{*}{U.D.} &  &  &  \\\hline
    \endfirsthead
    \hline
    RA1 & RA2 & RA3 & RA4 & RA5 & RA6 & U.D. & Título y descripción & D. & P. \\\hline
    \endhead
    
    \multicolumn{6}{| c |}{} &  & Total &  & 100\% \\\hline
    \endlastfoot
    
    X & & & & & & 1 & \makecell[l{X}]{\textbf{Fundamentos:}\\ Elementos fundamentales de un sistema operativo} & ? & 20\% \\\hline
    
    & X & & & & & 3 & \makecell[l{X}]{\textbf{Instalación:}\\ Proceso de instalación de varios sistemas operativos monousuario.} & 5 & 15\% \\\hline
    
    & & X & & & & 4 & \makecell[l{X}]{\textbf{Configuración:}\\ Configuración del sistema operativo adecuándolo a las necesidades del usuario en integrándolo en el entorno de trabajo.} & 5 & 20\% \\\hline
    
    & & & X & & & 5 & \makecell[l{X}]{\textbf{Administración y optimización:}\\ Donde trataremos de las tareas de administración y métodos de análisis de carga para la optimización del sistema.} & 5 & 10\% \\\hline
    
    & & & & X & & 6 & \makecell[l{X}]{\textbf{Herramientas de linea de comandos:}\\ Donde veremos el funcionamiento de la linea de comandos la sus herramientas administrativas principales.} & 5& 15\% \\\hline
    
    & & & & & X & 2 & \makecell[l{X}]{\textbf{Virtualización:}\\ Describimos las características y utilidad de las máquinas virtuales. Crearemos varias máquinas virtuales en las que instalaremos distintos sistemas operativos.} & 5& 20\% \\\hline
\end{tabularx}
%\end{sidewaystable}