\subsubsection{Identificación de la unidad didáctica}

\noindent
\needspace{5\baselineskip}
\begin{tabularx}{\textwidth}{c >{\centering\arraybackslash}X c}
    \toprule
    \thead{Nº} & \thead{Título de la U.D.} & \thead{Duración\\ (sesiones)}\\ \midrule
    1 & \TituloUD{1} & \NumSesionesUD{1}\\
    \bottomrule
\end{tabularx}


\subsubsection{Resultados de aprendizaje del currículo que se tratan}

\noindent
\needspace{3\baselineskip}
\begin{tabularx}{\linewidth}{X c}
    \toprule
    \thead{Resultados de aprendizaje del currículo} & \thead{Completo} \\ \midrule
    RA1. Reconoce los fundamentos y funciones de los sistemas operativos y los sistemas de ficheros, e identifica sus elementos & Sí \\
    \bottomrule    
\end{tabularx}


\subsubsection{Objetivos específicos de la unidad didáctica}

\bgroup
\rowcolors{4}{lightgray!25}{}
\needspace{5\baselineskip}
\begin{tabularx}{\linewidth}{X c X c}
    \toprule
    \thead{Objetivos específicos} & \thead{Act.} & \thead{Título de la actividad} & \thead{Duración\\ (sesiones)} \\ \midrule
    \endfirsthead
    \thead{Objetivos específicos} & \thead{Act.} & \thead{Título de la actividad} & \thead{Duración\\ (sesiones)} \\ \midrule
    \endhead
    O.E.1.1. Identificar las funciones de un S.O. & 1 & Funciones del S.O. & ?? \\
    O.E.1.2. Identificar las funciones de un sistema de ficheros & 2 & Funciones del sistema de ficheros & ?? \\
    \bottomrule
\end{tabularx}
\egroup

\subsubsection[Criterios de evaluación]{Criterios de evaluación que se aplicarán para la verificación de los objetivos por parte del alumnado}

\bgroup
\rowcolors{4}{lightgray!25}{}
\noindent
\needspace{5\baselineskip}
\begin{tabularx}{\linewidth}{X c c c}
    \toprule
    \thead{Criterios de evaluación} & \thead{Instrumentos\\ de evaluación} & \thead{Mínimos\\ exigibles} & \thead{Peso\\ cualificación} \\ \midrule
    \endhead
    % \makecell[cl]{CE1.1. Se identificaron y describie-\\ron los elementos funcionales de un\\ sistema informático} & PE & Sí & 15 \% \\ 
    CE1.1. Se identificaron y describieron los elementos funcionales de un sistema informático & PE & Sí & 20 \% \\
    CE1.2. Se codificó y relacionó la información en varios sistemas de representación & PE & No & 5 \% \\
    CE1.3. Se analizaron las funciones de los sistemas operativos & PE & Sí & 20 \% \\
    CE1.4. Se describió la arquitectura de los sistemas operativos & TO & Sí & 25 \% \\
    CE1.5. Se identificaron los procesos y sus estados & TO & No & 5 \% \\
    CE1.6. Se identificaron las posibilidades de partición de subsistema de almacenaje & LC & Sí & 15 \% \\
    CE1.7. Se describió la estructura y organización del sistema de ficheros & PE & Sí & 10 \% \\
    CE1.8. Se constató la utilidad de los sistemas transaccionales y su repercusión al seleccionar un sistema de ficheros & PE & No & 5 \% \\
    \bottomrule
\end{tabularx}
\egroup


\subsubsection{Contenidos}


\subsubsection[Actividades]{Actividades de ensañenaza y aprendizaje, y de evaluación, con justificación de para qué y de cómo se realizarán, así como los materiales y los recursos necesarios para su realización y, de ser el caso, los instrumentos de evaluación}
\begin{landscape}
    \bgroup
    \rowcolors{4}{lightgray!25}{}
    \noindent
    \needspace{5\baselineskip}
    \begin{tabularx}{\linewidth}{X X X X X X r}
        \toprule
        \thead{Qué es y\\ para qué} & \multicolumn{3}{c}{\thead{Cómo}} & \thead{Con qué} & \thead{Cómo es y\\ con qué se\\ valora} &  \\ \cmidrule{1-6}
        \thead{Actividad} & \thead{Profesorado\\ (tareas)} & \thead{Alumnado\\ (tareas)} & \thead{Materiales} & \thead{Resultados o\\ productos} & \thead{Instrumentos} & \multirowthead{-2}[5mm]{Duración\\ (sesiones)} \\
        \midrule
        \endfirsthead
        \thead{Actividad} & \thead{Profesorado} & \thead{Alumnado} & \thead{Materiales} & \thead{Resultados} & \thead{Instrumentos} & \thead{Duración\\ (sesiones)} \\
        \midrule
        \endhead
        \multicolumn{6}{r}{Total:} & XXX & \\ \bottomrule
        \endlastfoot
\end{tabularx}
\egroup
\end{landscape}