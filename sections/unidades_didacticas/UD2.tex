\subsection{\protect\TituloUD{2}}

\subsubsection{Identificación de la unidad didáctica}

\noindent
\needspace{5\baselineskip}
\begin{tabularx}{\linewidth}{c C c}
    \toprule
    \thead{Nº} & \thead{Título de la U.D.} & \thead{Duración\\(sesiones)}\\ \midrule
    2 & \TituloUD{2} & \NumSesionesUD{2}\\
    \bottomrule
\end{tabularx}


\subsubsection{Resultados de aprendizaje de currículo que se tratan}

\noindent
\needspace{3\baselineskip}
\begin{tabularx}{\linewidth}{X c}
    \toprule
    \thead{Resultados de aprendizaje del currículo} & \thead{Completo} \\ \midrule
    RA6. Utiliza máquinas virtuales, identifica su campo de aplicación e  instala software específico & Sí \\
    \bottomrule    
\end{tabularx}


\subsubsection{Objetivos específicos de la unidad didáctica}

\bgroup
\rowcolors{4}{lightgray!25}{}
\noindent
\needspace{5\baselineskip}
\begin{tabularx}{\linewidth}{X c X c}
    \toprule
    \thead{Objetivos específicos} & \thead{Act.} & \thead{Título de las activadades} & \thead{Duración\\(sesiones)}\\ \midrule
    OE2.1. Determinar las características fundamentales del proceso de virtualización & 1 & Introducción a la virtualización & 10 \\
    OE2.2. Conocer las distintas opciones de software de virtualización & 2 & Software de virtualizacíón & 10 \\ 
    OE2.3. Crear y configurar máquinas virtuales & 3 & Creación y configuración de máquinas virtuales & 20 \\ 
    \bottomrule
\end{tabularx}
\egroup

\subsubsection[Criterios de evaluación]{Criterios de evaluación que se aplicarán para la verificación de la consecución de los objetivos por parte del alumnado}

\bgroup
\rowcolors{4}{lightgray!25}{}
\begin{tabularx}{\linewidth}{X c c c}
    \toprule
    \thead{Criterios de evaluación} & \thead{Instrumentos\\ de evaluación} & \thead{Mínimos\\ exigibles} & \thead{Peso\\cualificación} \\ \midrule
    \endhead
    CE6.1. Se diferenció entre máquina real y máquina virtual & PE & Sí & 10 \% \\
    CE6.2. Se establecieron las ventajas y los inconvenientes del uso de máquinas virtuales & PE & Sí & 10 \% \\
    CE6.3. Se analizaron las principales herramientas para la creación y utilización de máquinas virtuales & TO & Sí & 5 \% \\
    CE6.4. Se instaló software libre y propietario para la creación y utilización de máquinas virtuales & LC & Sí & 10 \% \\
    CE6.5. Se crearon máquinas virtuales a partir de sistemas operativos libres y propietarios & LC & Sí & 15 \% \\
    CE6.6. Se configuraron máquinas virtuales & LC & Sí & 20 \% \\
    CE6.7. Se documentaron las opciones tomadas en la elección, instalación y configuración de la máquina virtual & LC & No & 5 \% \\
    CE6.8. Se relacionó la máquina virtual con el sistema operativo anfitrión & LC & No & 5 \% \\
    CE6.9. Se realizaron pruebas de rendimiento del sistema & LC & No & 5 \% \\
    \bottomrule
\end{tabularx}
\egroup


\subsubsection[Actividades de enseñanza, aprendizaje y evaluación; justificación, materiales y recursos]{Actividades de enseñanza y aprendizaje, y de evaluación, con justificación de para qué y de cómo se realizarán, así como los materiales y los recursos necesarios para su realización y, de ser el caso, los instrumentos de evaluación}