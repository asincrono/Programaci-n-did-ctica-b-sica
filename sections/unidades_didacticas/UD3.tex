\subsubsection{Identificación de la unidad didáctica}

\noindent
\needspace{5\baselineskip}
\begin{tabularx}{\linewidth}{c C c}
    \toprule
    \thead{Nº} & \thead{Título de la U.D.} & \thead{Duración\\(sesiones)}\\ \midrule
    3 & \TituloUD{3} & \NumSesionesUD{3}\\
    \bottomrule
\end{tabularx}


\subsubsection{Resultados de aprendizaje del currículo que se tratan}

\noindent
\needspace{3\baselineskip}
\begin{tabularx}{\linewidth}{X c}
    \toprule
    \thead{Resultados de aprendizaje del currículo} & \thead{Completo} \\ \midrule
    RA2. Instala sistemas operativos, para lo que consulta e interpreta la documentación técnica & Sí \\
    \bottomrule    
\end{tabularx}


\subsubsection{Objetivos específicos de la unidad didáctica}

\bgroup
\rowcolors{4}{lightgray!25}{}
\noindent
\needspace{5\baselineskip}
\begin{tabularx}{\linewidth}{X c X c}
    \toprule
    \thead{Objetivos específicos} & \thead{Act.} & \thead{Título de las activadades} & \thead{Duración\\(sesiones)}\\ \midrule
    OE3.1. Determinar requerimientos hardware específicos de los distintos S.O. & 1 & Análisis de requerimientos & 10 \\
    OE3.2. Elaborar un plan de instalación & 2 & Planificación de la instalación & 10 \\ 
    OE3.3. Instalar distintos S.O. & 3 & Instalación de S.O. & 20 \\ 
    \bottomrule
\end{tabularx}
\egroup


\subsubsection{Criterios de evaluación que se aplicarán para la verificación de la consecución de los objetivos por parte del alumnado}

\bgroup
\rowcolors{4}{lightgray!25}{}
\begin{tabularx}{\linewidth}{X c c c}
    \toprule
    \thead{Criterios de evaluación} & \thead{Instrumentos\\ de evaluación} & \thead{Mínimos\\ exigibles} & \thead{Peso\\cualificación} \\ \midrule
    \endhead
    CE2.1. Se analizó la documentación técnica del hardware para verificar su idoneidad & PE & Sí & 10 \% \\
    CE2.2. Se seleccionó el sistema operativo & TO & Sí & 5 \% \\
    CE2.3. Se elaboró un plan de instalación a partir de los manuales del sistema operativo & TO & No & 5 \% \\
    CE2.4. Se crearon y formatearon las particiones necesarias en los dispositivos de almacenamiento & LC & Sí & 15 \% \\
    CE2.5. Se configuraron los parámetros básicos de la instalación & TO & Sí & 10 \% \\
    CE2.6. Se instaló el sistema operativo & LC & Sí & 20 \% \\
    CE2.7. Se configuró un gestor de arranque & LC & No & 5 \% \\
    CE2.8. Se instalaron los controladores de dispositivos necesarios & LC & Sí & 15 \% \\
    CE2.9. Se documentaron las decisiones tomadas y las incidencias surgidas durante el proceso de instalación & LC & No & 5 \% \\ 
    CE2.10. Se respetaron las normas de utilización del software & PE & Sí & 5 \% \\
    CE2.11. Se trabajó con sistemas operativos libres y propietarios & TO & Sí & 15 \% \\
    \bottomrule
\end{tabularx}
\egroup


\subsubsection{Contenidos}

\begin{tabularx}{\linewidth}{X}
    \toprule
    \thead{Contenidos} \\ \midrule
    \textbf{BC2. Instalación de sistemas operativos libres y propietarios}\\
    \tabitem Requisitos técnicos del sistema operativo \\
    \tabitem Planificación de la instalación \\
    \tabitem Partición de los dispositivos de almacenamiento \\
    \tabitem Selección de un sistema de ficheros \\
    \tabitem Gestores de arranque \\
    \tabitem Tipos de instalación: típica y personalizada \\
    \tabitem Controladores de dispositivos \\
    \tabitem Selección de aplicaciones básicas para instalar \\
    \tabitem Parámetros básicos de la instalación \\
    \bottomrule
\end{tabularx}

\begin{landscape}
    \subsubsection[Actividades de ensañenaza y aprendizaje]{Actividades de ensañenaza y aprendizaje, y de evaluación, con justificación de para qué y de cómo se realizarán, así como los materiales y los recursos necesarios para su realización y, de ser el caso, los instrumentos de evaluación}
    
    \bgroup
    \rowcolors{3}{lightgray!25}{}
    \noindent
    \needspace{5\baselineskip}
    \begin{tabularx}{\linewidth}{p{0.13\linewidth} p{0.13\linewidth} p{0.13\linewidth} p{0.13\linewidth} p{0.13\linewidth} p{0.13\linewidth} r}
        \hiderowcolors
        \toprule
        \thead{Qué es y\\ para qué} & \multicolumn{3}{c}{\thead{Cómo}} & \thead{Con qué} & \thead{Cómo es y\\ con qué se\\ valora} &  \\ \cmidrule{1-6}
        \thead{Actividad} & \thead{Profesorado\\ (tareas)} & \thead{Alumnado\\ (tareas)} & \thead{Materiales} & \thead{Resultados o\\ productos} & \thead{Instrumentos} & \multirowthead{-2}[5mm]{Duración\\ (sesiones)} \\
        \midrule
        \endfirsthead
    
        \toprule
        \thead{Actividad} & \thead{Profesorado} & \thead{Alumnado} & \thead{Materiales} & \thead{Resultados} & \thead{Instrumentos} & \thead{Duración\\ (sesiones)} \\
        \midrule
        \endhead
    
        \midrule
        \endfoot
    
        \multicolumn{6}{r}{Total:} & ??? \\ \bottomrule
        \endlastfoot
    
        \showrowcolors
        
    \end{tabularx}
    \egroup
    \end{landscape}