\subsubsection{Identificación de la unidad didáctica}

\noindent
\needspace{5\baselineskip}
\begin{tabularx}{\linewidth}{c C c}
    \toprule
    \thead{Nº} & \thead{Título de la U.D.} & \thead{Duración\\(sesiones)}\\ \midrule
    4 & \TituloUD{4} & \NumSesionesUD{4}\\
    \bottomrule
\end{tabularx}


\subsubsection{Resultados de aprendizaje del currículo que se tratan}

\noindent
\needspace{3\baselineskip}
\begin{tabularx}{\linewidth}{X c}
    \toprule
    \thead{Resultados de aprendizaje del currículo} & \thead{Completo} \\ \midrule
    RA3. Realiza tareas básicas de configuración de sistemas operativos, para lo que interpreta requisitos, y describe los procedimientos seguidos & Sí \\
    \bottomrule    
\end{tabularx}


\subsubsection{Objetivos específicos de la unidad didáctica}

\bgroup
\rowcolors{4}{lightgray!25}{}
\noindent
\needspace{5\baselineskip}
\begin{tabularx}{\linewidth}{X c X c}
    \toprule
    \thead{Objetivos específicos} & \thead{Act.} & \thead{Título de las activadades} & \thead{Duración\\(sesiones)}\\ \midrule
    OE4.1. Conocer y distinguir los distintos tipos de interfaces de usuario & 1 & Interfaces de usuario & 10 \\
    OE4.2. Configurar las opciones básicas del entorno de usuario & 2 & Configuración básica & 10 \\ 
    OE4.3. Navegar por el sistema de ficheros así como crear, modificar y borrar elementos del mismo & 3 & Manejo del sistema de ficheros & 20 \\ 
    \bottomrule
\end{tabularx}
\egroup


\subsubsection{Criterios de evaluación que se aplicarán para la verificación de la consecución de los objetivos por parte del alumnado}

\bgroup
\rowcolors{4}{lightgray!25}{}
\noindent
\begin{tabularx}{\linewidth}{X c c c}
    \toprule
    \thead{Criterios de evaluación} & \thead{Instrumentos\\ de evaluación} & \thead{Mínimos\\ exigibles} & \thead{Peso\\cualificación} \\ \midrule
    \endhead
    CE3.1. Se diferenciaron las interfaces de usuario según sus propiedades & PE & Sí & 5 \% \\
    CE3.2. Se aplicaron preferencias en la configuración del entorno de usuario & LC & Sí & 10 \% \\
    CE3.3. Se gestionaron los sistemas de ficheros específicos & TO & Sí & 10 \% \\
    CE3.4. Se distinguieron los atributos de un fichero de los de un directorio & TO & Sí & 10 \% \\
    CE3.5. Se reconocieron los permisos de los ficheros y directorios & PE & Sí & 10 \% \\
    CE3.6. Se utilizaron los asistentes de configuración del sistema (acceso a redes, dispositivos, etc.) & LC & Sí & 10 \% \\
    CE3.7. Se comprobó la existencia de actualizaciones del sistema operativo y de los controladores de dispositivos & TO & Sí & 10 \% \\
    CE3.8. Se realizó la instalación de los parches del sistema operativo y de las versiones actuales de los controladores de dispositivos & TO & No & 5 \% \\
    CE3.9. Se realizó la configuración para la actualización periódica del sistema operativo & LC & Sí & 5 \% \\
    CE3.10. Se documentaron los procesos de actualización realizados sobre el sistema & TO & No & 2 \% \\  
    CE3.11. Se ejecutaron operaciones para la automatización de tareas del sistema & LC & No & 3 \% \\
    CE3.12. Se realizaron operaciones de instalación y desinstalación de utilidades & TO & Sí & 5 \% \\
    CE3.13. Se aplicaron métodos para la recuperación del sistema operativo & LC & Sí & 15 \% \\
    \bottomrule
\end{tabularx}
\egroup


\subsubsection{Contenidos}

\begin{tabularx}{\linewidth}{X}
    \toprule
    \thead{Contenidos}\\ \midrule
    \textbf{BC3. Realización de tareas básicas sobre sistemas operativos libres y propietarios:}\\
    1. Arranque y parada de sistema: sesiones\\
    2. Interfaces de usuario: tipos, propiedades y usos\\
    3. Utilización del sistema operativo: modo comando y modo gráfico\\
    4. Explotación del sistema operativo\\
    5. Configuración de las preferencias de escritorio\\
    6. Estructura del árbol de directorios\\
    7. Fichero y directorio: atributos y permisos\\
    8. Compresión y descompresión\\
    9. Actualización del sistema operativo y de los controladores de dispositivos\\
    10. Añadir, eliminar hardware del sistema operativo\\
    11. Agregar, eliminar y actualizar software del sistema operativo\\
    12. Operaciones de reparación del sistema operativo\\
    13. Configuración de la conexión a internet\\
    14. Programación de tareas\\
    \bottomrule
\end{tabularx}

\begin{landscape}
\subsubsection[Actividades de enseñanza, aprendizaje y evaluación; justifiCEción, materiales y recursos]{Actividades de enseñanza y aprendizaje, y de evaluación, con justificación de para qué y de cómo se realizarán, así como los materiales y los recursos necesarios para su realización y, de ser el caso, los instrumentos de evaluación}

\noindent    
\needspace{5\baselineskip}
\begin{tabularx}{\linewidth}{p{0.13\linewidth} p{0.13\linewidth} p{0.13\linewidth} p{0.13\linewidth} p{0.13\linewidth} p{0.13\linewidth} r}
    
    \toprule
    \thead{Qué es y\\ para qué} & \multicolumn{3}{c}{\thead{Cómo}} & \thead{Con qué} & \thead{Cómo es y\\ con qué se\\ valora} &  \\ \cmidrule{1-6}
    \thead{Actividad} & \thead{Profesorado\\ (tareas)} & \thead{Alumnado\\ (tareas)} & \thead{Materiales} & \thead{Resultados o\\ productos} & \thead{Instrumentos} & \multirowthead{-2}[5mm]{Duración\\ (sesiones)} \\
    \midrule
    \endfirsthead

    \toprule
    \thead{Actividad} & \thead{Profesorado} & \thead{Alumnado} & \thead{Materiales} & \thead{Resultados} & \thead{Instrumentos} & \thead{Duración\\ (sesiones)} \\
    \midrule
    \endhead

    \midrule
    \endfoot

    \multicolumn{6}{r}{Total:} & 34 \\ \bottomrule
    \endlastfoot

    A4.1. Interfaces de usuario & TP4.1.1 Exposición de los aspectos teóricos de los interfaces de usuario\newline TP4.1.2. Demostración práctica de la configuración del entorno personal\newline TP4.1.3. Presentación de TA4.1.1.\newline TP4.1.4. Corrección de PE4.1.1. y puesta en común de resultados & TA4.1.1. Realizar práctica de config. del entorno personal según requisitos\newline TA4.1.2. Reflejar TA4.1.1. en el diario de prácticas\newline TA4.1.3. Realización de PE4.1.1. & Hoja de requisitos para TA4.1.1.\newline PE4.1.1. impresa\newline PCs con máquinas virtuales de los S.S.O.O. & S.O. con entorno personal configurado\newline PE4.1.1. resuelta\newline Entrada en el diario de prácticas & LC y PE4.1.1. & 6\\

    A4.2. Gestión de ficheros & TP4.2.1. Exposición de aspectos teóricos de gestión del sist. de fich.\newline TP4.2.2. Presentación de TA4.2.1.\newline TP4.2.3. Corrección de PE4.2.1. y puesta en común& TA4.2.1. Práctica de gestión del sist. de fich. siguiendo requisitos\newline TA4.2.2. Realización de la PE4.2.1. de opción múltiple & Documento de requisitos\newline PE4.2.1. impresa\newline PC con máquinas virtuales del S.O. & PE4.2.1. resuelta\newline Entrada en el diario de prácticas & LC y PE4.2.1. & 6\\
    A4.3. Asistentes de configuración & TP4.3.1. Presentación de la tarea TA4.3.1. y TA4.3.2.\newline TP4.3.2. Orientación durante TA4.3.1. y TA4.3.2.\newline TP4.3.3. Moderar durante TA4.3.2. & TA4.3.1. Buscar en Internet información sobre el uso de distintos asistentes de configuración\newline TA4.3.2. Usar asistentes de configuración para configurar el sistema & Máquina virtual con S.O. & Entrada en el diario de prácticas & TO & 4\\

    A4.4. Actualización del sistema & TP4.4.1. Demostración práctica del proceso de actualización del sistema operativo\newline TP4.4.2. Demostración práctica del proceso de actualización de controladores\newline TP4.4.3. Presentación de TA4.4.1. y T4.4.2.\newline TP4.4.4 Moderar TA4.4.4.& TA4.4.1. Realizar una actualización del S.O. y controladores\newline TA4.4.2. Programar periodicidad de la actualización\newline TA4.4.3. Reflejar las acciones tomadas en el diario de prácticas\newline TA4.4.4. Puesta en común de lo aprendido & & & LC y TO & 7\\

    A4.5. Aplicaciones & TP4.5.1. Exposición teórico-práctica del proceso de instalación y desinstalación de aplicaciones\newline TP4.5.2. Presentación de TA4.5.1.\newline TP4.5.3. Exposición teórico-práctica de activación y desactivación de características del sistema\newline TP4.5.4. Presentación de TA4.5.2.\newline TP4.5.5. Moderación TA4.5.3. & TA4.5.1. Realizar un proceso de instalación y desintalación de aplicaciones\newline TA4.5.2. Realizar un proceso de activación y desactivación de características del S.O.\newline TA4.5.3. Conferir lo realizado al diario de prácticas\newline TA4.5.4. Puesta en común de lo aprendido & Documentos de requisitos para TA4.5.1. y TA4.5.2.\newline Máquinas virtuales con S.O. & Entradas en el diario de prácticas para TA4.5.1 y TA4.5.2. & LC y TO & 7\\

    A4.6. Automatización & TP4.6.1. Se ejecutaron operaciones para automatizar tareas\newline TP4.6.2. Presentación de TA4.6.1.\newline TP4.6.3. Moderar TA4.6.2.\newline TP4.6.4. Evaluación de TA4.6.1. mediante TO y diario de prácticas & TA4.6.1. Realización de tareas de automatización\newline TA4.6.2. Puesta en común de lo aprendido & & & LC y TO& 10\\

    A4.7. Recuperación del S.O. & TP4.7.1. & TA4.7.1. & & & TO & 5\\
\end{tabularx}
\end{landscape}
