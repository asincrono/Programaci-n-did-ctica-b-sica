\subsection{\protect\TituloUD{4}}

\subsubsection{Identificación de la unidad didáctica}

\noindent
\needspace{5\baselineskip}
\begin{tabularx}{\linewidth}{c C c}
    \toprule
    \thead{Nº} & \thead{Título de la U.D.} & \thead{Duración\\(sesiones)}\\ \midrule
    4 & \TituloUD{4} & \NumSesionesUD{4}\\
    \bottomrule
\end{tabularx}


\subsubsection{Resultados de aprendizaje del currículo que se tratan}

\noindent
\needspace{3\baselineskip}
\begin{tabularx}{\linewidth}{X c}
    \toprule
    \thead{Resultados de aprendizaje del currículo} & \thead{Completo} \\ \midrule
    RA3. Realiza tareas básicas de configuración de sistemas operativos, para lo que interpreta requisitos, y describe los procedimientos seguidos & Sí \\
    \bottomrule    
\end{tabularx}


\subsubsection{Objetivos específicos de la unidad didáctica}

\noindent
\needspace{5\baselineskip}
\begin{tabularx}{\linewidth}{X c X c}
    \toprule
    \thead{Objetivos específicos} & \thead{Act.} & \thead{Título de las activadades} & \thead{Duración\\(sesiones)}\\ \midrule
    OE4.1. Conocer y distinguir los distintos tipos de interfaces de usuario & 1 & Interfaces de usuario & 10 \\
    OE4.2. Configurar las opciones básicas del entorno de usuario & 2 & Configuración básica & 10 \\ 
    OE4.3. Navegar por el sistema de ficheros así como crear, modificar y borrar elementos del mismo & 3 & Manejo del sistema de ficheros & 20 \\ 
    \bottomrule
\end{tabularx}


\subsubsection{Criterios de evaluación que se aplicarán para la verificación de la consecución de los objetivos por parte del alumnado}

\bgroup
\rowcolors{4}{lightgray!25}{}
\noindent
\begin{tabularx}{\linewidth}{X c c c}
    \toprule
    \thead{Criterios de evaluación} & \thead{Instrumentos\\ de evaluación} & \thead{Mínimos\\ exigibles} & \thead{Peso\\cualificación} \\ \midrule
    \endhead
    CE3.1. Se diferenciaron las interfaces de usuario según sus propiedades & PE & Sí & 5 \% \\
    CE3.2. Se aplicaron preferencias en la configuración del entorno de usuario & LC & Sí & 10 \% \\
    CE3.3. Se gestionaron los sistemas de ficheros específicos & TO & Sí & 10 \% \\
    CE3.4. Se distinguieron los atributos de un fichero de los de un directorio & TO & Sí & 10 \% \\
    CE3.5. Se reconocieron los permisos de los ficheros y directorios & PE & Sí & 5 \% \\
    CE3.6. Se utilizaron los asistentes de configuración del sistema (acceso a redes, dispositivos, etc.) & LC & Sí & 10 \% \\
    CE3.7. Se comprobó la existencia de actualizaciones del sistema operativo y de los controladores de dispositivos & TO & Sí & 10 \% \\
    CE3.8. Se realizó la instalación de los parches del sistema operativo y de las versiones actuales de los controladores de dispositivos & Sí & 10 \% \\
    CE3.9. Se realizó la configuración para la actualización periódica del sistema operativo & Sí & 5 \% \\
    CE3.10. Se documentaron los procesos de actualización realizados sobre el sistema & No & TO & 2 \% \\  
    CE3.11. Se ejecutaron operaciones para la automatización de tareas del sistema & No & LC & 3 \% \\
    CE3.12. Se realizaron operaciones de instalación y desinstalación de utilidades & Sí & TO & 5 \% \\
    CE3.13. Se aplicaron métodos para la recuperación del sistema operativo & Sí & LC & 15 \% \\
    \bottomrule
\end{tabularx}
\egroup


\subsubsection{Contenidos}

\subsubsection[Actividades de enseñanza, aprendizaje y evaluación; justifiCEción, materiales y recursos]{Actividades de enseñanza y aprendizaje, y de evaluación, con justificación de para qué y de cómo se realizarán, así como los materiales y los recursos necesarios para su realización y, de ser el caso, los instrumentos de evaluación}

