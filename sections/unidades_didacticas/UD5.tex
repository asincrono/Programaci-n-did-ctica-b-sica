
\subsubsection{Identificación de la unidad didáctica}

\noindent
\needspace{5\baselineskip}
\begin{tabularx}{\linewidth}{c C c}
    \toprule
    \thead{Nº} & \thead{Título de la U.D.} & \thead{Duración\\(sesiones)}\\ \midrule
    5 & \TituloUD{5} & \NumSesionesUD{5}\\
    \bottomrule
\end{tabularx}


\subsubsection{Resultados de aprendizaje del currículo que se tratan}

\noindent
\needspace{3\baselineskip}
\begin{tabularx}{\linewidth}{X c}
    \toprule
    \thead{Resultados de aprendizaje del currículo} & \thead{Completo} \\ \midrule
    RA4. Realiza operaciones básicas de administración de sistemas operativos, para lo que interpreta requisitos, y prepara el sistema para su uso óptimo & Sí \\
    \bottomrule    
\end{tabularx}


\subsubsection{Objetivos específicos de la unidad didáctica}

\bgroup
\rowcolors{4}{lightgray!25}{}
\noindent
\needspace{5\baselineskip}
\begin{tabularx}{\linewidth}{X c X c}
    \toprule
    \thead{Objetivos específicos} & \thead{Act.} & \thead{Título de las activadades} & \thead{Duración\\(sesiones)}\\ \midrule
    OE4.1. Gestionar usuarios y grupos & 1 & Usuarios y grupos & 10 \\
    OE4.2. Conocer la estructura y organizar un sistema de ficheros & 2 & Organización del sistema de ficheros & 10 \\ 
    OE4.3. Listar los procesos de usuario, modificar su prioridad y eliminar procesos desbocados & 3 & Control de procesos & 20 \\ 
    \bottomrule
\end{tabularx}
\egroup


\subsubsection{Criterios de evaluación que se aplicarán para la verificación de la consecución de los objetivos por parte del alumnado}

\bgroup
\rowcolors{4}{lightgray!25}{}
\noindent
\begin{tabularx}{\linewidth}{X c c c}
    \toprule
    \thead{Criterios de evaluación} & \thead{Instrumentos\\ de evaluación} & \thead{Mínimos\\ exigibles} & \thead{Peso\\cualificación} \\ \midrule
    \endhead
    CE3.1. Se diferenciaron las interfaces de usuario según sus propiedades & PE & Sí & 5 \% \\
    CE3.2. Se aplicaron preferencias en la configuración del entorno de usuario & LC & Sí & 10 \% \\
    CE3.3. Se gestionaron los sistemas de ficheros específicos & TO & Sí & 10 \% \\
    CE3.4. Se distinguieron los atributos de un fichero de los de un directorio & TO & Sí & 10 \% \\
    CE3.5. Se reconocieron los permisos de los ficheros y directorios & PE & Sí & 5 \% \\
    CE3.6. Se utilizaron los asistentes de configuración del sistema (acceso a redes, dispositivos, etc.) & LC & Sí & 10 \% \\
    CE3.7. Se comprobó la existencia de actualizaciones del sistema operativo y de los controladores de dispositivos & TO & Sí & 10 \% \\
    CE3.8. Se realizó la instalación de los parches del sistema operativo y de las versiones actuales de los controladores de dispositivos & Sí & 10 \% \\
    CE3.9. Se realizó la configuración para la actualización periódica del sistema operativo & Sí & 5 \% \\
    CE3.10. Se documentaron los procesos de actualización realizados sobre el sistema & No & TO & 2 \% \\  
    CE3.11. Se ejecutaron operaciones para la automatización de tareas del sistema & No & LC & 3 \% \\
    CE3.12. Se realizaron operaciones de instalación y desinstalación de utilidades & Sí & TO & 5 \% \\
    CE3.13. Se aplicaron métodos para la recuperación del sistema operativo & Sí & LC & 15 \% \\
    \bottomrule
\end{tabularx}
\egroup



\subsubsection{Contenidos}

\begin{tabularx}{\linewidth}{X}
    \toprule
    \thead{Contenidos}\\ \midrule
    \textbf{BC4. Administración de los sistemas operativos}\\
    1. Gestión de perfiles de usuarios y grupos locales\\
    2. Gestión del sistema de ficheros\\
    3. Herramientas para la gestión de ficheros y directorios\\
    4. Gestión de los procesos del sistema y de usuario\\
    5. Rendimiento del sistema. Seguimiento de la actividad del sistema\\
    6. Activación y desactivación de servicios\\
    7. Compartición de recursos\\
    8. Gestión de dispositivos de almacenamiento\\
    9. Gestión de impresoras\\
    10. Base de datos de configuración y comportamiento del sistema operativo, del hardware instalado y las aplicaciones\\
    \bottomrule
\end{tabularx}
\subsubsection[Actividades de enseñanza, aprendizaje y evaluación; justificación, materiales y recursos]{Actividades de enseñanza y aprendizaje, y de evaluación, con justificación de para qué y de cómo se realizarán, así como los materiales y los recursos necesarios para su realización y, de ser el caso, los instrumentos de evaluación}

