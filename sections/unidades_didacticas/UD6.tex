\subsubsection{Identificación de la unidad didáctica}

\noindent
\needspace{5\baselineskip}
\begin{tabularx}{\linewidth}{c C c}
    \toprule
    \thead{Nº} & \thead{Título de la U.D.} & \thead{Duración\\(sesiones)}\\ \midrule
    6 & \TituloUD{6} & \NumSesionesUD{6}\\
    \bottomrule
\end{tabularx}


\subsubsection{Resultados de aprendizaje del currículo que se tratan}

\noindent
\needspace{3\baselineskip}
\begin{tabularx}{\linewidth}{X >{\centering\arraybackslash}m{2.5cm}} 
    \toprule
    \thead{Resultados de aprendizaje del currículo} & \thead{Completo} \\ \midrule
    RA5. Realiza tareas de gestión y administración básica de sistemas operativos, para lo que utiliza herramientas en línea de comandos & Sí \\
    \bottomrule    
\end{tabularx}


\subsubsection{Objetivos específicos de la unidad didáctica}
\rowcolors{4}{lightgray!25}{}
\noindent
\needspace{5\baselineskip}
\begin{tabularx}{\linewidth}{X c X c}
    \toprule
    \thead{Objetivos específicos} & \thead{Act.} & \thead{Título de las activadades} & \thead{Duración\\(sesiones)}\\ 
    \midrule
    \endfirsthead
    \thead{Objetivos específicos} & \thead{Act.} & \thead{Título de las activadades} & \thead{Duración\\(sesiones)}\\ 
    \midrule
    \endhead
    OE6.1. Identificar las características de los interfaces de linea de comandos así como sus pautas de uso & 1 & Introducción a la línea de comandos & 10 \\
    OE6.2. Interactuar con el sistema de ficheros mediante la linea de comandos & 3 & Comandos del sistema de ficheros & 10 \\ 
    OE6.3. Utilizar los comandos básicos del sistema operativo & 2 & Comandos básicos & 20 \\ 
    \bottomrule
\end{tabularx}


\subsubsection{Criterios de evaluación que se aplicarán para la verificación de la consecución de los objetivos por parte del alumnado}

\bgroup
\rowcolors{4}{lightgray!25}{}
\noindent
\needspace{5\baselineskip}
\begin{tabularx}{\linewidth}{X c c c}
    \toprule
    \thead{Criterios de evaluación} & \thead{Instrumentos\\ de evaluación} & \thead{Mínimos\\ exigibles} & \thead{Peso\\ cualificación} \\ \midrule
    \endfirsthead
    \thead{Criterios de evaluación} & \thead{Instrumentos\\ de evaluación} & \thead{Mínimos\\ exigibles} & \thead{Peso\\ cualificación} \\ \midrule
    \endhead
    % \makecell[cl]{CE1.1. Se identificaron y describie-\\ron los elementos funcionales de un\\ sistema informático} & PE & Sí & 15 \% \\ 
    CE5.1. Se identificaron las diferencias entre las interfaces y las interfaces en línea de comandos & PE & Sí & 5 \% \\
    CE5.2. Se estableció el modo de acceso a la consola y las pautas para su uso & PE & Sí & 10 \% \\
    CE5.3. Se describieron las capacidades generales de los intérpretes de comandos y las posibilidades de selección, en función del sistema operativo & PE & Sí & 10 \% \\
    CE5.4. Se utilizaron comandos para actuar sobre ficheros y directorios & TO & Sí & 15 \% \\
    CE5.5. Se utilizaron otros comandos habituales propios del sistema operativo & TO & Sí & 15 \% \\
    CE5.6. Se aplicaron redirecciones sobre la entrada y la salida de los comandos & TO & Sí & 10 \% \\
    CE5.7. Se aplicaron opciones para modificar el comportamiento de los comandos & TO & Sí & 10 \% \\
    CE5.8. Se accedió a la ayuda para obtener información sobre la utilización de los comandos & TO & No & 5 \% \\ 
    \bottomrule
\end{tabularx}
\egroup


\subsubsection{Contenidos}

\begin{tabularx}{\linewidth}{X}
    \toprule
    \thead{Contenidos}\\ \midrule
    \textbf{BC5. Utilización de la línea de comandos}\\
    1. Características de los intérpretes de comandos\\
    2. Utilización de órdenes para la gestión de ficheros y directorios\\
    3. Operación con ficheros: nombre y extensión, comodines, atributos y tipos\\
    4. Operación con directorios: nombre, atributos y permisos\\
    5. Otras operaciones comunes en la linea de comandos\\
    6. Redirección de la entrada y la salida\\
    7. Activación de opciones en los comandos\\
    8. Utilización de la ayuda en línea\\
    \bottomrule
\end{tabularx}


\subsubsection[Actividades de enseñanza, aprendizaje y evaluación; justificación, materiales y recursos]{Actividades de enseñanza y aprendizaje, y de evaluación, con justificación de para qué y de cómo se realizarán, así como los materiales y los recursos necesarios para su realización y, de ser el caso, los instrumentos de evaluación}


\bgroup
\rowcolors{3}{lightgray!25}{}
\noindent
\needspace{5\baselineskip}
\begin{tabularx}{\linewidth}{p{0.13\linewidth} p{0.13\linewidth} p{0.13\linewidth} p{0.13\linewidth} p{0.13\linewidth} p{0.13\linewidth} r}
    \hiderowcolors
    \toprule
    \thead{Qué es y\\ para qué} & \multicolumn{3}{c}{\thead{Cómo}} & \thead{Con qué} & \thead{Cómo es y\\ con qué se\\ valora} &  \\ \cmidrule{1-6}
    \thead{Actividad} & \thead{Profesorado\\ (tareas)} & \thead{Alumnado\\ (tareas)} & \thead{Materiales} & \thead{Resultados o\\ productos} & \thead{Instrumentos} & \multirowthead{-2}[5mm]{Duración\\ (sesiones)} \\
    \midrule
    \endfirsthead

    \toprule
    \thead{Actividad} & \thead{Profesorado} & \thead{Alumnado} & \thead{Materiales} & \thead{Resultados} & \thead{Instrumentos} & \thead{Duración\\ (sesiones)} \\
    \midrule
    \endhead

    \midrule
    \endfoot

    \multicolumn{6}{r}{Total:} & 27 \\ \bottomrule
    \endlastfoot

    \showrowcolors
    
\end{tabularx}
\egroup