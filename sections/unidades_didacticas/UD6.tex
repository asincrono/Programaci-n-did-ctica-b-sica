\subsection{\protect\TituloUD{6}}

\subsubsection{Identificación de la unidad didáctica}

\noindent
\needspace{5\baselineskip}
\begin{tabularx}{\linewidth}{c C c}
    \toprule
    \thead{Nº} & \thead{Título de la U.D.} & \thead{Duración\\(sesiones)}\\ \midrule
    6 & \TituloUD{6} & \NumSesionesUD{6}\\
    \bottomrule
\end{tabularx}


\subsubsection{Resultados de aprendizaje del currículo que se tratan}

\noindent
\needspace{3\baselineskip}
\begin{tabularx}{\linewidth}{X >{\centering\arraybackslash}m{2.5cm}} 
    \toprule
    \thead{Resultados de aprendizaje del currículo} & \thead{Completo} \\ \midrule
    RA5. Realiza tareas de gestión y administración básica de sistemas operativos, para lo que utiliza herramientas en línea de comandos & Sí \\
    \bottomrule    
\end{tabularx}


\subsubsection{Objetivos específicos de la unidad didáctica}
\subsubsection{Criterios de evaluación que se aplicarán para la verificación de la consecución de los objetivos por parte del alumnado}
\subsubsection{Contenidos}
\subsubsection[Actividades de enseñanza, aprendizaje y evaluación; justificación, materiales y recursos]{Actividades de enseñanza y aprendizaje, y de evaluación, con justificación de para qué y de cómo se realizarán, así como los materiales y los recursos necesarios para su realización y, de ser el caso, los instrumentos de evaluación}